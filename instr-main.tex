%% 
%% Copyright 2007-2025 Elsevier Ltd
%% 
%% This file is part of the 'Elsarticle Bundle'.
%% ---------------------------------------------
%% 
%% It may be distributed under the conditions of the LaTeX Project Public
%% License, either version 1.3 of this license or (at your option) any
%% later version.  The latest version of this license is in
%%    http://www.latex-project.org/lppl.txt
%% and version 1.3 or later is part of all distributions of LaTeX
%% version 1999/12/01 or later.
%% 
%% The list of all files belonging to the 'Elsarticle Bundle' is
%% given in the file `manifest.txt'.
%% 
%% Template article for Elsevier's document class `elsarticle'
%% with harvard style bibliographic references

\documentclass[preprint,12pt,authoryear]{elsarticle}

\usepackage[utf8]{inputenc}
\usepackage[margin=2cm]{geometry}
\usepackage{graphicx}
\usepackage{multirow}
\usepackage{amssymb}
\usepackage{amsmath}
\usepackage{setspace}
\usepackage{outlines}
\usepackage{enumitem}
\usepackage{xcolor}
\usepackage{upgreek}
\usepackage{mathabx}
\usepackage{algorithm}
\usepackage{algorithmic}
\usepackage{amsthm}
\usepackage[labelfont=bf,font=small]{caption}
\usepackage{epsfig}
\usepackage{geometry}
\usepackage{subfigure}
\usepackage[textsize=tiny]{todonotes}
\usepackage[normalem]{ulem}
\usepackage{lipsum}
\usepackage{array}
\usepackage{booktabs}
\usepackage{lineno}
\usepackage{array}
\usepackage{tikz}
\usepackage[english]{babel}
\usepackage{animate}
\usepackage{cancel}

\usepackage[]{siunitx}
\sisetup{range-units=single,separate-uncertainty = true,print-unity-mantissa=false,per-mode=symbol,range-phrase = \text{--},
inter-unit-product=\cdot
}

\usepackage[
    protrusion=true,
    activate={true,nocompatibility},
    final,
    tracking=true,
    kerning=true,
    spacing=true,
    factor=1100]{microtype}
    
\SetTracking{encoding={*}, shape=sc}{40}

\usepackage{outlines}
\usepackage{enumitem}

\definecolor{lightblue}{rgb}{0.63, 0.74, 0.78}
\definecolor{seagreen}{rgb}{0.18, 0.42, 0.41}
\definecolor{orange}{rgb}{0.85, 0.55, 0.13}
\definecolor{silver}{rgb}{0.69, 0.67, 0.66}
\definecolor{rust}{rgb}{0.72, 0.26, 0.06}
\definecolor{purp}{RGB}{68, 14, 156}

\definecolor{zblue}{RGB}{8,81,156}
\definecolor{zpurp}{RGB}{84,39,143}
\definecolor{zred}{RGB}{165,15,21}

\colorlet{lightrust}{rust!50!white}
\colorlet{lightorange}{orange!25!white}
\colorlet{lightlightblue}{lightblue}
\colorlet{lightsilver}{silver!30!white}
\colorlet{darkorange}{orange!75!black}
\colorlet{darksilver}{silver!65!black}
\colorlet{darklightblue}{lightblue!65!black}
\colorlet{darkrust}{rust!85!black}
\colorlet{darkseagreen}{seagreen!85!black}



\usepackage{hyperref}
\hypersetup{
  colorlinks=true,
}
\usepackage{tabularx}
\usepackage{bbm}
\usepackage{bm}

\usepackage[nameinlink]{cleveref}
\crefname{equation}{}{}
\def\appendixname{}
\crefname{appendix}{}{}


\usepackage{setspace}
% \doublespacing
\setlength{\heavyrulewidth}{1.5pt}
% \setlength{\abovetopsep}{4pt}
 
\usepackage{soul}
\sethlcolor{yellow}

\usepackage[parfill]{parskip}

\usepackage{lineno}
\usepackage{tcolorbox}
%\linenumbers

%% Self-defined Macros
\newcommand{\bn}[1]{\mathbf{#1}}
\newcommand{\bs}[1]{\ensuremath{\boldsymbol{#1}}}
\newcommand{\mc}[1]{\ensuremath{\mathcal{#1}}}
\newcommand{\norm}[1]{\ensuremath \lVert #1 \rVert}
\newcommand{\GD}[1]{ D #1 (\boldsymbol{F})[\boldsymbol{H}]}
\newcommand{\GDI}[1]{ D #1 (\boldsymbol{I})[\boldsymbol{H}]}
\newcommand{\Lin}[1]{ L_{\boldsymbol{F}} #1 [\boldsymbol{H}]}
\newcommand{\LinI}[1]{ L_{\boldsymbol{I}} #1 [\boldsymbol{H}]}
\newcommand{\Dm}[1]{\ensuremath{\boldsymbol{\varphi} } }
\newcommand{\tr}[1]{\textcolor{red}{#1}}
\newcommand{\ignore}[1]{}
\newcommand{\fracp}[2]{\frac{\partial #1}{\partial #2}}
\newcommand{\wt}[1]{\widetilde{#1}}
\renewcommand{\[}{\left[}
\renewcommand{\]}{\right]}
\renewcommand{\(}{\left(}
\renewcommand{\)}{\right)}
\newcommand{\tn}{\textnormal}
\newcommand{\gradX}{\nabla_{\bm{X}}}
\newcommand{\gradx}{\nabla_{\bm{x}}}
\newcommand{\gradF}{\nabla_{\bn{F}}}

%\newcommant{\pfrac[2]}{\frac{\partial {#1}}{\partial {#2}}}

%% Coloring for comments
\usepackage{color,tikz,soul}
\newcommand{\jbe}[1]{\textcolor{violet}{#1}}

\newcommand{\rev}[1]{\textcolor{red}{#1}}

\usepackage{xcolor, soul}
\sethlcolor{yellow}
\usepackage[per-mode=symbol]{siunitx}

\setlength{\tabcolsep}{12pt}

\renewcommand{\thetable}{\arabic{table}} % Originally Roman uppercase

%% Use the option review to obtain double line spacing
%% \documentclass[authoryear,preprint,review,12pt]{elsarticle}

%% Use the options 1p,twocolumn; 3p; 3p,twocolumn; 5p; or 5p,twocolumn
%% for a journal layout:
%% \documentclass[final,1p,times,authoryear]{elsarticle}
%% \documentclass[final,1p,times,twocolumn,authoryear]{elsarticle}
%% \documentclass[final,3p,times,authoryear]{elsarticle}
%% \documentclass[final,3p,times,twocolumn,authoryear]{elsarticle}
%% \documentclass[final,5p,times,authoryear]{elsarticle}
%% \documentclass[final,5p,times,twocolumn,authoryear]{elsarticle}

%% For including figures, graphicx.sty has been loaded in
%% elsarticle.cls. If you prefer to use the old commands
%% please give \usepackage{epsfig}

%% The amssymb package provides various useful mathematical symbols
%\usepackage{amssymb}
%% The amsmath package provides various useful equation environments.
%\usepackage{amsmath}
%% The amsthm package provides extended theorem environments
%% \usepackage{amsthm}

%% The lineno packages adds line numbers. Start line numbering with
%% \begin{linenumbers}, end it with \end{linenumbers}. Or switch it on
%% for the whole article with \linenumbers.
%% \usepackage{lineno}

\journal{ME517---Mechanics of Soft Materials}

\setlength{\marginparwidth}{2cm}
\begin{document}

\begin{frontmatter}

\title{MECHENG 517---Mechanics of Soft Materials} %% Article title

%% Make a new file, e.g. "main.tex", and then replace my name with yours!
\author{Prof. Jon Estrada} 


\affiliation{organization={University of Michigan},%Department and Organization
            addressline={2350 Hayward St.}, 
            city={Ann Arbor},
            postcode={48105}, 
            state={MI},
            country={USA}}

%% Abstract
% \begin{abstract}
% %% Text of abstract
% Abstract text.
% \end{abstract}

%%Graphical abstract
% %\includegraphics{grabs}
% \begin{graphicalabstract}
% \end{graphicalabstract}

%%Research highlights
% \begin{highlights}
% \item Research highlight 1
% \item Research highlight 2
% \end{highlights}

%% Keywords
% \begin{keyword}
%% keywords here, in the form: keyword \sep keyword

%% PACS codes here, in the form: \PACS code \sep code

%% MSC codes here, in the form: \MSC code \sep code
%% or \MSC[2008] code \sep code (2000 is the default)

% \end{keyword}

\end{frontmatter}

%% If you want to include line numbers, uncomment the line below this:
% \linenumbers
\section*{Preparation for Oral Assessment I}

Each student will be given one of three topics in viscoelasticity described below.  
Your oral assessment will be a maximum of 20 minutes long in Prof. Estrada's office, GGB 2450. 
You may prepare and bring a \textbf{single slide (pptx or pdf) per topic}, one of which you will use as a visual aid. 

The goal of this assessment is to demonstrate and communicate your conceptual understanding of viscoelasticity, as well as your ability to reason aloud. 

\begin{itemize}
\item[\textbf{1. }]\textbf{Possible Topics} 
\begin{outline} 
\1 Comparative behavior of analog models under different loading scenarios 
\1 Given some data, predicting material response to a given loading function 
\1 Superposition and the requirements for linear viscoelasticity 
\end{outline}
\item[\textbf{2. }]\textbf{Your Slides} 
\begin{outline} 
\1 Your slide should be maximally helpful in guiding you in each topic, and may include diagrams, equations, and short bullet points (defined as being fewer than 2 lines of text per bullet). 
\1 The minimum text size is 16-point font. 
\1 Text must be in one of Arial, Times, Helvetica, Palatino, or Computer Modern (with others with approval of instructor).
\end{outline}
\item[\textbf{3. }]\textbf{Presentation} 
\begin{outline}
\1 You will open with a (timed) \textbf{3-minute-maximum} explanation, using your slide if desired to walk through any key concepts and their importance. 
\1 After your explanation, you will enter a Q\&A/interactive discussion with Prof. Estrada focused on a specific scenario and further reasoning with hypothetical variations on the scenario. 
\1 You should be able to define key terms, interpret the physical meaning of models, and contextualize material physics within soft material behavior. 
\1 You are encouraged to “think aloud”—explaining your reasoning is what I'm looking for, even if you make corrections as you go. 
\end{outline}
\end{itemize}



\newpage


% Additional notes sections for ME517

%\include{instr-mathnotation}


% \include{instr-part6}
% \include{instr-part5}
% \include{instr-part4}
\section*{Project III: Data Gathering and Summary (Due Oct 17)}

Step three of your whitepaper development is the evidence gathering stage. 
Before proposing to investigate something that takes significant time and resources, you must demonstrate that there is a promising lead.  
For this milestone, your task is to \textbf{find, select, and curate} datasets from literature, open datasets, or your own work if available, that are pertinent to your identified gap from the previous step.

Note that you are \textbf{not} expected or required to perform analysis at this stage---that will be the next project checkpoint. 
Instead, focus on actually finding these datasets, understanding what they tell us, and briefly contextualizing these results in relation to both the original study and your own project. 

You'll execute this particular aim as a Powerpoint slide deck of 4-6 slides, saved down as a pdf. \textit{A strong submission will contain all of the following:}


\renewcommand{\outlinei}{itemize}
\begin{enumerate}
\item[\textbf{0.}] \textbf{Title Slide}
\begin{outline}
\1 Project working title, your name, a one-sentence restatement of your research gap. Please use the \hyperlink{https://me.engin.umich.edu/student-intranet/}{ME department template}. 
\end{outline}
\item[\textbf{1.}] \textbf{Brief description of your project}
\begin{outline}
\1 Single sentence summaries of the following:
\2 The state of the field around your topic of interest
\2 The gap in that field
\2 What your project broadly would aim to achieve%A visual summary (table or schematic) listing all data sources (with citation).
\2 General description of the data that you found, and why it's relevant to that aim
\end{outline}
\item[\textbf{2--6.}] \textbf{Individual Dataset Slides}
\begin{outline}
\1 For each selected dataset (aim for 3--5 distinct datasets including both computational and experimental works), include:
\2 Figure/table of data (may be reproduced from literature with citation, or created from open databases/your own data)
\2 Bullet points containing:
\3 Some detail about what the plot of table shows, including context (e.g. how it was generated, under what circumstances, and what variables were controlled-for)
\3 Full citation for the work
\3 Why this dataset is relevant for your proposed project and gap
\end{outline}
\end{enumerate}


%This is just a placeholder for now

%\include{instr-part2}
%\include{instr-part1}
%\section*{Proposing a Research Topic in the Mechanics of Soft Materials}

As your major deliverable this semester, you are going to develop a ``white paper'' proposal for a research project in the area of soft material mechanics. 
This is deliberately a bit different from a standard report by focusing on being forward-looking. 
A white paper is a $3-5$ page document that contains:
\begin{enumerate}
    \item A short \textbf{overview} of your topic
    \item A brief summary of the \textbf{current}, or state-of-the-art of \textbf{knowledge} in that area
    \item The \textbf{knowledge gap} you are interested in filling
    \item The \textbf{long-term goal} of what your project would do in the field if successful
    \item The testable \textbf{central hypothesis} governing the work
    \item Three \textbf{specific aims} you would pursue with their own working hypotheses
    \item \textbf{Preliminary data} backing up why you formulated and believe those hypotheses
    \item The \textbf{expected outcomes} or products of your proposed research that link back to the specific aims  
\end{enumerate}

This project development will occur over the course of the entire semester, and every assignment you have this semester will have a component that will develop this project proposal in some way. 
The schedule will be as follows:
\smallskip

\footnotesize
\begin{tabularx}{\textwidth}{ccXX}

\textbf{Checkpoint} & \textbf{Due Date} & \textbf{Major Component(s)} & \textbf{White Paper Section(s)} \\
\hline
\hline
P1 & Sept 19 & Topic ID and overview & Overview, long-term goal \\
\hline
P2 & Oct 3 & Literature review & Current knowledge and gap \\
\hline
P3 & Oct 17 & Data gathering and summary & Preliminary data \\
\hline
P4 & Nov 7 & Data analysis and & Central hypothesis \\
 &  & central hypothesis & Support from preliminary data \\
\hline
P5 & Nov 21 & Three specific aims and & Specific aims \\
 & & supported working hypotheses & Expected outcomes \\
\hline
P6 & Dec 8 & Full white paper draft & All sections \\
 &  & (and peer review) &  \\
\hline
-- & Dec 15--16 & Oral presentations & Final draft \\
\hline
\end{tabularx}
\normalsize

%% The appendix will contain the example problems assigned as problem sets. 
\appendix

%include{instr-PS5}
%\include{instr-PS4}
\setcounter{section}{3} % This causes the next section to be Appendix B


\section*{Examples III. Linear Viscoelastic Models}
\label{PS3}

This set of example problems is due on October 17, 2025. 

% This is a placeholder for the example problems from the third problem set. 
% You'll replace this file with the one I supply on canvas. 

\medskip
\subsection*{3--1. \textbf{Converting creep to relaxation} [4 pts].} 
Say we measure the creep function for a material by fitting a sum of exponential functions to some data. 
We determine the creep function to be 
\begin{equation}
    J_c(t) = \frac{1}{1000}\left(10 - 5 e^{-t/4} - 3e^{-t/8}\right).
\end{equation}
(a) Attach a plot of $J_c(t)$, labeling significant values.

(b) Determine the corresponding stress relaxation function $G_r(t)$. What are the characteristic stress relaxation times now, and how do they compare to the creep relaxation times? 

\bigskip
\bigskip
\bigskip
\subsection*{3--2. \textbf{Alternate standard linear solid model} [4 pts].}

In class, we derived the relaxation and creep compliance functions $G_r(t)$ and $J_c(t)$ for a standard linear solid (SLS) model consisting of a spring in parallel with a Maxwell branch. 
In this question, we'll investigate a variant arrangement for the SLS, where a spring is placed in series with a Kelvin-Voigt solid. 

(a) Determine the differential constitutive law for the variant SLS. 

(b) Then, the creep compliance function $J_c(t)$ and hence, the relaxation function $G_r(t)$.

(c) How do the coefficients in the two variants of the standard solid model relate to each other?

\bigskip
\bigskip
\bigskip
\subsection*{3--3. \textbf{Frequency response of a 5-term analog model} [4 pts].}
You have a five-parameter fit $G_r(t) = C_r (200 e^{-2t} + 100 e^{-t} + 10)$ that describes the relaxation behavior of a real material. 

(a) Draw the equivalent mechanical analog model for this fit.

(b) Determine the functional forms for the storage and loss moduli, and create a semi-log plot of the loss tangent $(\tan\delta)$ over a domain of relevant frequency orders $(\log \omega)$. 

\newpage
\subsection*{3--4. \textbf{Fractional response} [4 pts].}

This question will be best approached numerically, using e.g. Matlab or Mathematica. 

Fractional order models can be used to show relaxation that does not follow the classic ``S-curve'' Debye relaxation function for $G_r(t)$ vs. $\log t$. 

Starting from a Kelvin-Voigt-type fractional model with the functional form of 
\begin{equation*}
    G_r(t) = \left[10 + 2\left(\frac{t}{0.2} \right)^{-\alpha}\right] \mathcal{H}(t),
\end{equation*}
plot the stress and strain responses of this solid over time (i.e., plot $\sigma(t,\alpha)$ and $\varepsilon(t,\alpha)$ on separate plots for each part) for a range of values of $0<\alpha<1$ to (a) a step strain, (b) a step stress of only length $t=5$, and (c) another stress function entirely of your choice. 

As a suggestion, you could consider values spaced symmetrically around zero on the logistic distribution, which is defined as $\textrm{logit}(\alpha) = \log\left(\frac{\alpha}{1-\alpha} \right)$. 
Picking e.g., logit($\alpha$)$=0$ corresponds to $\alpha =0.5$, $\textrm{logit}(\alpha) =  1$ is $\alpha\approx0.73$, etc. 
I suggest sampling integers on a range of logit($\alpha$) $= -4 \textrm{~to~} 4$ to cover the full range from elastic to viscous response for the springpot.

\bigskip
\bigskip
\subsection*{3--5. \textbf{Rheology without a rheometer} [8 pts].}

You have a rubbery material of density $\rho$ for which you plan to characterize frequency-dependent viscoelastic behavior. 
The material you have can be made into a sphere of a wide range of sizes, from a radius of $R=1$ mm to $R=1$ m. 
You plan to drop each ball onto a rigid half-space from a height $h_0$, and can measure the rebound height $h(R)$ for each ball radius $R$. 

The impact duration for an elastic material is given by a Hertzian contact relation of
\begin{equation*}
    t_c = 5.21\frac{R}{c}\left(\frac{c}{\sqrt{2 g h_0}}\right)^{1/5} \approx 0.025R  \textrm{~~[s]}
\end{equation*}
where $c = 1000$ m/s represents the pressure wave speed in the material and the initial height $h_0$ is taken to be a consistent 0.01 m.

(a) How much energy per volume is dissipated by the material for each size of ball? 

(b) Using the Lissajous plot of $\sigma/|E^*|$ vs. $\varepsilon$, show that the approximate peak elastic energy stored in the ball during a half-cycle is $\frac{1}{2} B^2 \cos \delta$, where $B = \varepsilon_{\textrm{max}}$. 

(c) Determine an approximate expression for the energy dissipated by the ball during a drop event in terms of $A = \varepsilon_{\textrm{max}} \sin \delta$ and $B$. 

(d) Hence, determine $\tan\delta$ as a function of the rebound height, $h(R)$. 

(e) For what frequencies could you say this material is calibrated?
%\setcounter{section}{1} % This causes the next section to be Appendix B


\section*{Examples II: Kinetics, Constitutive Laws, and Viscoelasticity I}
\label{PS2}
\textcolor{red}{(Rev note: v3)}


This set of example problems is due on October 3, 2025. 
As before, I request that you type up your responses in \LaTeX~ rather than write them out by hand. 

\medskip
\subsection*{2--1. \textbf{Balance of mass} [4 pts].} 
A large piece of polydimethylsiloxane (PDMS) of uniform density $\rho(\bm{x},t)$ has a central spherical bubble of time-evolving radius $R(t)$, initial radius $R_0$, and wall velocity of $\dot{R}$. 
The hole is subject to a uniform surface traction in the $\bm{e}^{(r)} \equiv \bm{e}_{\bm{r}}$ direction from an axisymmetric pressure, and maintains spherical symmetry over time. 

\medskip
The position of a point in the material can be written as $\bm{x} = r(R,t) \bm{e}_{\bm{r}}$ with reference position $\bm{X} = r_0(R_0)$, while the velocity of that point can be written as $\bm{v}(r,t) = v_r \bm{e}_{\bm{r}}$.

\medskip
Using the conservation of mass equation, show that the material must satisfy
%\begin{equation}
%\rho_{,t} + (\rho v_i)_{,i} = 0,
%\end{equation}
\begin{equation*}
\rho_{,t}+ \rho_{,r} v_r + \frac{\rho}{r} (\textcolor{red}{2}v_r + r v_{r,r}) = 0,
\end{equation*}
and hence, show that an assumption of incompressibility for PDMS results in 
\begin{equation*}
v_r(r,t) = \frac{R^2 \dot{R}}{r^2}.
\end{equation*}

\medskip
\subsection*{2--2. \textbf{Balance of momenta} [4 pts].} A spherical hydrogel body $\mathcal{B}$ with a linear density gradient is currently submerged in water as depicted in the figure. 
The sphere has coordinates $\bm{x}$ in a region $\Omega$ with position-dependent density $\rho(\bm{x})$. 

\begin{figure}[H]
\vspace{-2em}
\centering
\includegraphics[width=3in]{instr-figures/PS2-Q1.pdf}
\caption{\small{Hydrogel sphere with a linear density gradient submerged in water. The water has a density $\rho_w$, while the sphere has a density at its leftmost point of $\rho_w/2$ and at its rightmost point of $3\rho_w/2$.}}
\end{figure}

\vspace{-1em}
The surface traction $\bm{t}(\bm{x},\hat{\bm{n}})$ acting on $\mathcal{B}$ is given by 
\begin{equation*}
\bm{t}(\bm{x},\hat{\bm{n}}) = -\rho_w g x_3 \hat{\bm{n}},
\end{equation*}
where $\hat{\bm{n}}$ is the outer unit normal to the surface $\partial \Omega_t$ and $\rho_w$ is the (constant) density of water and $g$ is the acceleration due to gravity. 

\medskip
(a) Determine the net force and moment acting on $\mathcal{B}$ via volume integrals.

\medskip
(b) Under what \textit{two} conditions is $\mathcal{B}$ in static equilibrium?


\bigskip
\subsection*{2--3. \textbf{Viscoelastic data} [4 pts].} 
Stress relaxation \textcolor{red}{(Ask yourself: does it actually matter whether it's stress relaxation or creep compliance?)} isochrones for a compliant viscoelastic material are shown in the figure below.  

\begin{figure}[H]
\vspace{-1em}
\centering
\includegraphics[scale = 1.5]{instr-figures/PS2-Q3.pdf}
\caption{\small{Stress (Pa) vs. strain ($-$) for a soft viscoelastic material.}}
\end{figure}

\vspace{-1em}
(a) Are these isochrones from a material which we can describe with linear viscoelasticity? If not, why not, and if so, under what approximate regimes would this assumption be valid? 

\medskip
(b) Estimate the creep relaxation function $J_c$ for stress values of 100 and 250 \textcolor{red}{Pa}. Isochrones are shown at times of 2, 5, 10, 20, and 40 seconds.   
% This is a placeholder for the example problems from the second problem set. 
% You'll replace this file with the one I supply on canvas. 

\bigskip
\subsection*{2--4. \textbf{Impulsive stresses} [4 pts].}

(a) Say that instead of a step load, we apply $\sigma(t) = A \delta(t)$ to an unknown linear viscoelastic material. 
Determine the strain history $\epsilon(t)$, first as a general function of the creep relaxation function $J_c(t)$, and then for a Kelvin-Voigt solid. 

(b) Now, consider a rapid load followed by a rapid reverse load by applying a doublet function of stress, i.e. $\sigma(t) = B \psi(t)$. 
What is the strain function $\epsilon(t)$ in terms of $J_c(t)$ and for a Kelvin-Voigt material now? 


\bigskip
\subsection*{2--5. \textbf{Two-element models} [8 pts].}

Dynamic mechanical analysis (DMA) is a common technique for characterizing viscoelasticity. 
DMA conventionally involves application of a sinusoidal displacement to the top surface of a sample at a controllable temperature. 
Often, the user puts the sample into initial compression, and follows with the sinusoidal profile. 
A cylindrical sample of height $h$ and diameter $d$ is placed between two plates.
The DMA then quickly puts the sample into compression by moving its top plate downward by a displacement $d$, and then oscillates sinusoidally between positions $0$ and $2d$ at a frequency $\omega$.

\medskip
(a) Using the constitutive law for a Kelvin-Voigt material, determine the stress $\sigma(t)$ exerted by the platens to cause the applied strain. 

\medskip
(b) The resulting stress lags behind the strain by an phase $\delta$, as in $\sin(\omega t + \delta)$. 
Commonly this is reported as the ``tangent loss'', or $\tan\delta$, for a material. 
What is the value of $\tan\delta$ for this particular Kelvin-Voigt model?

\medskip
(c) Say instead of prescribing the strain $\epsilon(t)$, we instead prescribed the stress, $\sigma(t) = - \sigma_0 - \sigma_0 \sin(\omega t)$. 
Determine the strain $\epsilon(t)$ for this prescribed stress.

\medskip
(d) Prove that the tangent loss function $\tan\delta$ is identical between the two loading methods.






%\include{instr-PS1}
%\setcounter{section}{1} 
\section*{Solutions to Examples I. Mathematical Preliminaries}
\label{soln-PS1}

1--1. If the strain function is
\begin{equation*}
    \varepsilon(t) = \int_0^t J(t-\tau) \frac{d\sigma(\tau)}{d\tau} d\tau,
\end{equation*}
then we need to write the Laplace transform as 
\begin{equation*}
    \mathcal{L}\{ \varepsilon(t) \} = \mathcal{L} \left\{ \frac{d\sigma(t)}{dt} \right\} \cdot \mathcal{L}\left\{ J(t)\right\}.
\end{equation*}
For both cases we'll use the compliance function $J(t) = J_\infty + (J_0-J_\infty)\exp[-t/\tau_c]$ but we'll try two different stress terms, $\sigma_1(t) = \sigma_0 H(t)$ and $\sigma_2(t) = \sigma_0  \sin(\omega t)$. 

Case (a) proceeds as follows:
\begin{align*}
    \mathcal{L}\{ \varepsilon_1(t) \} &= \mathcal{L}\left\{ \frac{d}{dt}\left(\sigma_0 \mathcal{H}(t)\right) \right\} \cdot \mathcal{L}\left\{ J(t) \right\}\\
        &= \mathcal{L}\left\{\sigma_0 \delta(t) \right\} \cdot \mathcal{L}\left\{ J_\infty + (J_0-J_\infty)\exp[-t/\tau_c] \right\}\\
        &= \sigma_0 \cdot \left[ J_\infty \mathcal{L}\{ 1\} + (J_0 - J_\infty)\mathcal{L}\{\exp[-t/\tau_c] \}\right] \\
    \mathcal{L}\{ \varepsilon_1(t) \} &= \sigma_0 \left[ \frac{J_\infty}{s} + \frac{J_0 - J_\infty}{s+1/\tau_c} \right] 
\end{align*}

Case (b) is similar, but requires us to carry out the time-derivative of $\sigma_0 \sin(\omega t)$, which is $\sigma_0 \omega\cos(\omega t)$. 
This changes the pre-factor to result in:
\begin{equation*}
     \mathcal{L}\{ \varepsilon_2(t) \} = \omega \sigma_0 \frac{s}{s^2+\omega^2} \left[ \frac{J_\infty}{s} + \frac{J_0 - J_\infty}{s+1/\tau_c} \right].
\end{equation*}

Now to complete \textit{dare mode}, we use the following rule about residues of simple (i.e., linear) poles of a function $\bar{f}(s)$ multiplied by $e^{st}$:
\begin{equation*}
    \textrm{Residue of a pole of} ~\bar{f}(s) \textrm{: }\lim\limits_{s\rightarrow s_0} \left[(s-s_0) \bar{f}(s) \right]. 
\end{equation*}
For taking the inverse Laplace transform of the first strain function $\varepsilon_1(t)=\mathcal{L}^{-1}\{\bar{\varepsilon}_1(s)\}$, we need to find the sum of the residues of of all poles of $\bar{\varepsilon}_1(s)$. 
The first term of $\bar{\varepsilon}_1(s)$ has a pole at $s_0=0$, leading to $\sigma_0 J_\infty e^{0\cdot t} = \sigma_0 J_\infty$. 
The second term has a pole at $s_0 = -1/\tau_c$, resulting in a pole residue of $\sigma_0 (J_0 - J_\infty) e^{-t/\tau_0}$. 
In sum, the first strain relation gives
\begin{equation*}
        \varepsilon_1(t) = \sigma_0 \left[J_\infty + (J_0 - J_\infty)e^{-t/\tau_c} \right]. 
\end{equation*}

The second is slightly more complicated due to the additional complex roots in the two terms. 
\begin{equation*}
     \mathcal{L}\{ \varepsilon_2(t) \} = \frac{\omega \sigma_0 J_\infty}{(s+i\omega)(s-i\omega)}  +  \frac{\omega \sigma_0 s (J_0 - J_\infty)}{(s+1/\tau_c)(s+i\omega)(s-i\omega)} \equiv \bar{f}_1(s) + \bar{f}_2(s).
\end{equation*}
The inverse Laplace transform is then the sum of the five total residues of the poles, which is calculated as:
\begin{align*}
    \varepsilon_2(t) &= \mathcal{R}(\bar{f}_1(i \omega) e^{i \omega t}) + \mathcal{R}(\bar{f}_1(-i \omega) e^{-i \omega t}) + \mathcal{R}(\bar{f}_2(-1/\tau_c) e^{- t/\tau_c}) + \mathcal{R}(\bar{f}_2(i \omega) e^{i \omega t}) + \mathcal{R}(\bar{f}_2(-i \omega) e^{-i \omega t})\\
    &= \frac{\omega \sigma_0 J_\infty e^{i \omega t}}{2 i \omega} - \frac{\omega \sigma_0 J_\infty e^{-i \omega t}}{2 i \omega}+ \omega \sigma_0 (J_0 - J_\infty) \left[ \frac{(-\frac{1}{\tau_c}) e^{-t/\tau_c}}{\frac{1}{\tau_c^2}+\omega^2}+\frac{i \omega e^{i \omega t}}{(i\omega + \frac{1}{\tau_c})(2 i \omega)} + \frac{-i \omega e^{-i \omega t}}{(-i\omega + \frac{1}{\tau_c})(-2 i \omega)} \right],
\end{align*}
which simplifies if you combine the complex exponential terms into sinusoids into a slightly better final form of:
\begin{equation*}
    \varepsilon_2(t) = \sigma_0 J_\infty \sin\omega t + \sigma_0(J_0 -J_\infty)\frac{\omega}{\frac{1}{\tau_c^2}+\omega^2} \left[  -\frac{1}{\tau_c} e^{-t/\tau_c}+ \cos\omega t + \omega \sin \omega t \right].  
\end{equation*}

\subsection*{1--2. \textbf{Index notation} [4 pts].} 

\medskip
(a) $\bm{p} \times (\bm{q} \times \bm{r}) = (\bm{r} \cdot \bm{p}) \bm{q} - (\bm{q} \cdot \bm{p}) \bm{r}$.
\begin{align*}
    \bm{p} \times (\bm{q} \times \bm{r}) \rightarrow& ~~~~\epsilon_{ijk} p_j (\epsilon_{abc} q_b r_c )_k\\
    & =\epsilon_{kij} \epsilon_{kbc} p_j q_b r_c\\
    &=(\delta_{ib} \delta_{jc} - \delta_{ic}\delta_{jb} ) p_j q_b r_c\\
    &=p_i p_c r_c - r_i p_j q_j \rightarrow (\bm{p} \cdot \bm{r})\bm{q} - (\bm{p} \cdot \bm{q}) \bm{r}.//
\end{align*}

\medskip
(b) $(\bm{p} \times \bm{q}) \cdot (\bm{a} \times \bm{b}) = (\bm{p} \cdot \bm{a}) (\bm{q} \cdot \bm{b}) - (\bm{q} \cdot \bm{a})(\bm{p} \cdot \bm{b})$
\begin{align*}
    (\bm{p} \times \bm{q}) \cdot (\bm{a} \times \bm{b}) \rightarrow & ~~~~\epsilon_{ijk} p_j q_k \epsilon_{imn} a_m b_n\\
    & =\epsilon_{ijk} \epsilon_{imn} p_j q_k a_m b_n\\
    &=(\delta_{jm}\delta_{kn}-\delta_{jn}\delta_{km}) p_j q_k a_m b_n\\
    &= p_j q_k a_j b_k - p_j q_k a_k b_j\\
    &= p_j a_j q_k b_k - q_k a_k p_j b_j \rightarrow (\bm{p} \cdot \bm{a}) (\bm{q} \cdot \bm{b}) - (\bm{q} \cdot \bm{a})(\bm{p} \cdot \bm{b}).//
\end{align*}

\medskip
(c) $(\bm{a} \otimes \bm{b})(\bm{p} \otimes \bm{q}) = \bm{a}\otimes\bm{q}(\bm{b} \cdot \bm{p}) $
\begin{align*}
   (\bm{a} \otimes \bm{b})(\bm{p} \otimes \bm{q}) \rightarrow & ~~~~(a_i b_j \bm{e}_i \otimes \bm{e}_j) \cdot (p_k q_\ell \bm{e}_k \otimes \bm{e}_\ell) \\
    & = a_i b_j p_k q_\ell (\bm{e}_i \otimes \bm{e}_j) \cdot (\bm{e}_k \otimes \bm{e}_\ell)\\
    &= a_i b_j p_k q_\ell \delta_{jk} (\bm{e}_i \otimes \bm{e}_{\ell})\\
    &= a_i q_\ell  b_j p_j (\bm{e}_i \otimes \bm{e}_{\ell}) \rightarrow \bm{a}\otimes\bm{q}(\bm{b} \cdot \bm{p}).//
\end{align*}

\medskip
(d) $\bn{Q}^\intercal\bm{a} \cdot \bn{Q}^\intercal\bm{b} = \bm{a}\cdot\bm{b} $
\begin{align*}
   \bn{Q}^\intercal\bm{a} \cdot \bn{Q}^\intercal\bm{b} \rightarrow & ~~~~(Q^\intercal_{ij} a_j \bm{e}_i) \cdot (Q^\intercal_{k\ell} b_\ell \bm{e}_k) \\
    & = Q_{ji} a_j Q_{\ell k} b_\ell \delta_{ik}\\
    &= Q_{ji} a_j Q_{\ell i} b_\ell\\
    &= a_j Q_{ji} Q^\intercal_{i\ell} b_\ell ~~~~~\textrm{recall: }\bn{Q} \bn{Q}^\intercal = \bn{I}\\
    &= a_j \delta_{j\ell} b_\ell\\
    &= a_j b_j \rightarrow \bm{a} \cdot \bm{b}.//
\end{align*}

\bigskip
\subsection*{1--3. \textbf{Tensors and vectors} [4 pts].}

We want to show that $\bm{u} = (\bm{u} \cdot \bm{n}) \bm{n} + \bm{n} \times (\bm{u} \times \bm{n} )$ for a choice of vector $\bm{u}$ and unit vector $\bm{n}$. 

We can start with that $\bm{u} = \bn{I} \bm{u}$, which is written in index notation as $u_i = \delta_{ij} u_j$. 
From there we can add and subtract $n_i n_j$ to further manipulate the expression, i.e. 
\begin{equation*}
    u_i = (\delta_{ij} + n_i n_j - n_i n_j) u_j.
\end{equation*}
If the vector $\bm{n}$ is a unit vector, which has to be true for the projection tensor definitions to hold, then we can rearrange the above to
\begin{equation*}
    u_i = n_i n_j u_j + (\delta_{ij} - n_i n_j) u_j.
\end{equation*}
The first term is $(\bm{u}\cdot\bm{n})\bm{n}$. 
The second is the one we need to connect with $\bm{n} \times (\bm{u} \times \bm{n}$. 
Expanding the cross product using the result from the above problem 1--2(a), we have $(\bm{n} \cdot \bm{n})\bm{u} - (\bm{n}\cdot\bm{u})\bm{n}$, or in index notation, $n_j n_j u_i - n_j u_j n_i$. 
This simplifies to just $u_i - n_j u_j n_i$ from the unit length $n_i n_i =1$, which is indeed identical to the second term we have above. 

\bigskip
\subsection*{1--4. \textbf{Vector and tensor calculus} [4 pts].}

\medskip
(a) $\gradX \times (\phi \bm{a}) = \phi \gradX \times \bm{a} + (\gradX\phi) \times \bm{a}$
\begin{align*}
    \gradX \times (\phi \bm{a}) \rightarrow & ~~~~ \epsilon_{ijk} \frac{\partial}{\partial X_j} (\phi a_k)\\
    &= \epsilon_{ijk} ( \phi a_{k,j} + \phi_{,j} a_k) \rightarrow \phi \gradX \times \bm{a} + (\gradX\phi) \times \bm{a}.//
\end{align*}

\medskip
(b) $\gradX (\bm{a} \cdot \bm{b}) = (\bm{a} \cdot \gradX) \bm{b} + (\bm{b} \cdot \gradX) \bm{a} + \bm{a} \times (\gradX \times \bm{b}) + \bm{b} \times (\gradX \times \bm{a})$
It helps to go in with a strategy on indices here. 
I try, for example, to make every free index an $i$, and use $j$ for all of my first dummy indices. 
When implementing alternators and Kronecker deltas later, I try to prioritize $i-j-k$ over any other indices. 
\begin{align*}
    \gradX (\bm{a} \cdot \bm{b}) \rightarrow & ~~~~\frac{\partial}{\partial X_i}(a_j b_j)\\ 
    &= a_{j,i} b_j + a_j b_{j,i}.
\end{align*}
Now let's manipulate the right side of the equation to see what we can glean from the other terms. 
The first two are:
\begin{align*}
    (\bm{a} \cdot \gradX) \bm{b} + (\bm{b} \cdot \gradX) \bm{a} \rightarrow & ~~~~ a_j b_{i,j} + b_j a_{i,j}.
\end{align*}
The latter two are:
\begin{align*}
    \bm{a} \times (\gradX \times \bm{b}) + \bm{b} \times (\gradX \times \bm{a}) \rightarrow & ~~~~ \epsilon_{ijk} a_j \epsilon_{kqr} b_{r,q} + \epsilon_{ijk} b_j \epsilon_{kqr} a_{r,q}\\
    &= \epsilon_{kij} \epsilon_{kqr} a_j b_{r,q} + \epsilon_{kij} \epsilon_{kqr} b_j a_{r,q}\\
    &= (\delta_{iq} \delta_{jr}- \delta_{ir} \delta_{jq}) (a_j b_{r,q} + b_j a_{r,q})\\
    &= a_j b_{j,i} + b_j a_{j,i} - a_j b_{i,j} - b_j a_{i,j}.
\end{align*}
Note that the negative terms in this set cancel with the first two terms from the RHS, which leaves exactly what we have on the LHS. //

(c) $ (\bn{A} \bn{B}) \bn{:} \bn{C} = (\bn{A}^\intercal \bn{C})\bn{:} \bn{B} = (\bn{C} \bn{B}^\intercal)\bn{:} \bn{A}$.

This is one that's good to do with unit vector directions. 
\begin{align*}
     (\bn{A} \bn{B}) \bn{:} \bn{C} &= (A_{ij} B_{jk} \bm{e}_i \otimes \bm{e}_k) : (C_{mn} \bm{e}_m \otimes \bm{e}_n)\\
     &= A_{ij} B_{jk} \delta_{im} \delta_{kn} C_{mn}\\
     &= A_{ij} B_{jk} C_{ik} = (A^\intercal_{ji} C_{ik})B_{jk} = (C_{ik} B^\intercal_{kj}) A_{ij}.//
\end{align*}

(d) $\frac{\partial J}{\partial \bn{F}} = J \bn{F}^{-\intercal}$.

First, let's use the identity:
\begin{equation}
\label{eq:J}
    J = \frac{1}{6} \epsilon_{ijk} \epsilon_{pqr} F_{ip} F_{jq} F_{kr},
\end{equation}
where $J = \det \bn{F}$. 
Now, let's actually execute the derivative of $J$ with respect to $\bn{F}$:
\begin{equation*}
    \frac{\partial J}{\partial F_{mn}} = \frac{1}{6} \epsilon_{ijk} \epsilon_{pqr} (\delta_{im} \delta_{pn} F_{jq} F_{kr} + F_{ip} \delta_{jm} \delta_{qn} F_{kr} + F_{ip} F_{jq} \delta_{km} \delta_{rn}). 
\end{equation*}
Reindexing and circ-shifting the alternators makes all three of these terms actually the same, so we can write this quantity instead as
\begin{equation}
\label{eq:dJdFmn}
    \frac{\partial J}{\partial F_{mn}} = \frac{1}{2} \epsilon_{mjk} \epsilon_{nqr} F_{jq} F_{kr}.
\end{equation}
From here, it seems plausibly useful to try and isolate J on the RHS of the initial equation, which we can achieve by post-multiplying by $\bn{F}^\intercal$:
\begin{equation*}
    \frac{\partial J}{\partial \bn{F}} \bn{F}^\intercal = J \bn{F}^{-\intercal} \bn{F}^\intercal = J\bn{I}
\end{equation*}
From here, it's useful to write this in index notation:
\begin{equation*}
\frac{\partial J}{\partial F_{mn}} F^\intercal_{nk} = J \delta_{mk}
\end{equation*}
For our next trick, we can take the trace of this expression to get something that's a scalar like in our original identity expression:
\begin{equation}
\label{eq:dJdFmnFt}
    \frac{\partial J}{\partial F_{mn}} F^\intercal_{nm} = \frac{\partial J}{\partial F_{mn}} F_{mn} = J \delta_{kk} = 3J.
\end{equation}
We can combine equations \ref{eq:dJdFmn} and \ref{eq:dJdFmnFt} by post-multiplying the former by $\bn{F}^\intercal$:
\begin{equation*}
    \frac{\partial J}{\partial F_{mn}} F_{mn} = \frac{1}{2} \epsilon_{mjk} \epsilon_{nqr} F_{jq} F_{kr} F_{mn},
\end{equation*}
which does indeed equal $3J$ via equation \ref{eq:J}. //

\bigskip
\subsection*{1--5. \textbf{Kinematics} [8 pts].}

(a) Determine the deformation gradient tensor $[\bn{F}(\bm{X})]^{\bm{e}}$ for all $\bm{X}\in \mathcal{G}$. 
Describe any assumptions you make about the shape of the HGC as it deforms. 

The Happy Gelatinous Cube (HGC) seems to have sides that bend in an approximately quadratic way, while the top plane simply moves in an oscillatory fashion (and the bottom plane stays put). 
When getting a deformation field, it's often a good idea to start with either the mapping function or the displacement to ground your intuition. 
I'll proceed here with displacements. 

The clearest displacement component is the $u_2$ component, which most straightforwardly is a linear function that goes from zero at the bottom surface to $\alpha sin(\omega t)$ at the top surface. 
This can be written as
\begin{equation*}
    u_2(\bm{X}) =  X_2 \frac{\alpha}{2} \sin\omega t 
\end{equation*}
Note that the vertical displacement does not change as a function of $X_1$ or $X_3$, because all flat square sections remain flat during all deformations.
Analyzing points at the mid-plane wall helps us determine the displacements in the other two Cartesian directions. 
These must also be zero at the top and bottom surfaces, suggesting a quadratic function that is maximal at the middle. 
Lastly, the transverse displacements can reasonably be assumed to be zero at the middle and linearly increase to the outside. 
In all, the displacement components $u_1$ and $u_3$ are:
\begin{align*}
    u_1(\bm{X}) &= -X_1 X_2 (2-X_2)\beta \sin\omega t\\
    u_3(\bm{X}) &= -X_3 X_2 (2-X_2)\beta \sin\omega t,
\end{align*}
where the negative sign in front corresponds to the cube contracting in its middle when extended at the top.

The deformation gradient tensor $\bn{F} = \bn{I} + \gradX \bm{u}$ is then given in component form as:

\begin{equation*}
\bn{F} = \begin{bmatrix}
 1 - X_2 (2-X_2)\beta \sin\omega t & -X_1 (2-2X_2) \beta \sin\omega t & 0 \\
 0 & 1 + \frac{\alpha}{2} \sin\omega t & 0 \\
 0 & -X_3 (2-2X_2) \beta \sin\omega t & 1- X_2 (2-X_2)\beta \sin\omega t
 \end{bmatrix}. 
\end{equation*}

(b) Determine the stretch magnitude of a small fiber positioned at a height $X_2 = 1$ and oriented at an angle $\theta$ from the $\bm{e}_1$ axis (in either the $\bm{e}_1- \bm{e}_2$ or $\bm{e}_1- \bm{e}_3$ plane).

At the midplane, we have a small fiber pointed in, say, the $\hat{\bm{n}} = \cos\theta \bm{e}_1 + \sin\theta\bm{e}_3$ direction. 

The stretch of this fiber comes from $\lambda(\hat{\bm{n}}) = \sqrt{\hat{\bm{n}}\cdot \bn{C} \hat{\bm{n}}} = \sqrt{\bn{F} \hat{\bm{n}} \cdot \bn{F} \hat{\bm{n}}}$. 
Note that at the midplane, the shear terms of $\bn{F}$ end up being zero due to the $(2-2X_2)$ term, leaving the diagonal stretch terms of $\{1-\beta \sin \omega t$, $1 + \frac{\alpha}{2} \sin \omega t$, and $1-\beta \sin \omega t\}$ to be post-multiplied by $\hat{\bm{n}}$. 
This post-multiplication yields vectors of $\bn{F} \hat{\bm{n}} = 1-\beta \sin \omega t\{\cos \theta, 0, \sin\theta\}$, which then result in a total stretch of the prefactor $\lambda(\hat{\bm{n}}, X_2=1) = 1-\beta \sin \omega t$.

(c) Determine the Lagrange-Green strain tensor $\bn{E}$ and the material logarithmic strain tensor $\bn{E}_H = \ln (\bn{U})$ for the geometric center $\bm{X}_c$ of the HGC. What are the maximum and minimum values of the strain eigenvalues $E_i(t)$ and $E_i^H(t)$? 
Would you expect one set to be more symmetric about zero as $\alpha$ gets large, and why?

The deformation gradient tensor $\bn{F}$ at the center of the HGC is:
\begin{equation*}
\bn{F} = \begin{bmatrix}
 1 - \beta \sin\omega t & 0 & 0 \\
 0 & 1 + \frac{\alpha}{2} \sin\omega t & 0 \\
 0 & 0 & 1- \beta \sin\omega t
 \end{bmatrix}. 
\end{equation*}

Thus, the logarithmic strain is just the log of each of the entries of this diagonal tensor, i.e.,
\begin{equation*}
\bn{E}_H = \begin{bmatrix}
 \ln(1 - \beta \sin\omega t) & 0 & 0 \\
 0 & \ln(1 + \frac{\alpha}{2} \sin\omega t) & 0 \\
 0 & 0 & \ln(1- \beta \sin\omega t)
 \end{bmatrix}, 
\end{equation*}
with eigenvalues that oscillate between $\{\ln(1 + \frac{\alpha}{2}), \ln(1 - \beta), \ln(1 - \beta)\}$ and $\{\ln(1 - \frac{\alpha}{2}), \ln(1 + \beta), \ln(1 + \beta)\}$. 

The Lagrange-Green strain is given as $\bn{E} = \frac{1}{2}(\bn{F}^\intercal \bn{F} - \bn{I})$, which yields:
\begin{equation*}
\bn{E} = \frac{1}{2}\begin{bmatrix}
 (1 - \beta \sin\omega t)^2-1 & 0 & 0 \\
 0 & (1 + \frac{\alpha}{2} \sin\omega t)^2-1 & 0 \\
 0 & 0 & (1- \beta \sin\omega t)^2-1
 \end{bmatrix}. 
\end{equation*}
The extreme eigenvalues of $\bn{E}$ alternate between $\{-\beta+\frac{1}{2}\beta^2, \frac{\alpha}{2} + \frac{\alpha^2}{8}, -\beta+\frac{1}{2}\beta^2 \}$ and $\{\beta+\frac{1}{2}\beta^2, -\frac{\alpha}{2} + \frac{\alpha^2}{8}, \beta+\frac{1}{2}\beta^2 \}$. 

If we take $\alpha$ to be $4 \beta$, which approximates incompressible behavior, and make $\beta$ somewhat large, say, $0.4$---the values are, e.g. $\bn{E}_{Hi} = \{-.51, .59, -.51 \}$ and $ \{.34,-1.6, .34\}$. 
Compare that to $\bn{E}_i = \{-.32, 1.12, -.32 \}$ and $ \{.48,-.48, .48\}$.
The result suggests that the extrema are differently exaggerated depending on the strain metric, which makes sense---the domain of Lagrange-Green strain is $(-\frac{1}{2}, \infty)$, whereas that of the logarithmic strain is $(-\infty, \infty)$. 

d) Determine the material point acceleration $\bm{A}(\bm{X}_1)$ at a point $\frac{1}{2} \bm{e}_1 + 2\bm{e}_2 + \frac{1}{2} \bm{e}_3$.  

For this, we use the displacement 
\begin{equation*}
    \bm{u} =  -X_1 X_2 (2-X_2)\beta \sin\omega t \bm{e}_1 + \frac{\alpha}{2} X_2 \sin\omega t \bm{e}_2  -X_3 X_2 (2-X_2)\beta \sin\omega t \bm{e}_3,
\end{equation*}
and take two time derivatives: 
\begin{equation*}
    \bm{A} =  \omega^2 X_1 X_2 (2-X_2)\beta \sin\omega t \bm{e}_1 - \omega^2 X_2 \frac{\alpha}{2} \sin\omega t \bm{e}_2 +\omega^2 X_3 X_2 (2-X_2)\beta \sin\omega t \bm{e}_3.
\end{equation*}
Now we just plug in values of $\bm{X_1}$ corresponding to the point $\bm{X_1} = \frac{1}{2} \bm{e}_1 + 2\bm{e}_2 + \frac{1}{2} \bm{e}_3$. As expected, the horizontal displacement is zero on the top due to the boundary condition, so the total Lagrangian acceleration at that point is: 
\begin{equation*}
    \bm{A}(\bm{X_1}) =   - 2 \omega^2 \frac{\alpha}{2} \sin\omega t \bm{e}_2.// 
\end{equation*}


% %\newpage
% \bigskip
% \subsection*{1--5. \textbf{Kinematics} [8 pts].} The Happy Gelatinous Cube (HGC, pictued) $\mathcal{G}$ exists on a domain of $\{-1\leq X_1 , X_3\leq1, 0\leq X_2 \leq 2\}$ at initial time $t=0$. 
% At all times, the bottom surface of the HGC does not move. 
% Its top surface moves sinusoidally in time at frequency $\omega$ by a maximum magnitude of $\alpha$. 
% At maximum compression, points in the centers of the surfaces defined by outward normals $\bm{e}_1$ and $\bm{e}_3$ experience maximum displacements of magnitude $\beta$. 

% \medskip
% (a) Determine the deformation gradient tensor $[\bn{F}(\bm{X})]^{\bm{e}}$ for all $\bm{X}\in \mathcal{G}$. 
% Describe any assumptions you make about the shape of the HGC as it deforms. 

% \medskip
% (b) Determine the stretch magnitude of a small fiber positioned at a height $X_2 = 1$ and oriented at an angle $\theta$ from the $\bm{e}_1$ axis \textcolor{red}{(in either the $\bm{e}_1- \bm{e}_2$ or $\bm{e}_1- \bm{e}_3$ plane)}. 

% \medskip
% (c) Determine the Lagrange-Green strain tensor $\bn{E}$ and the material logarithmic strain tensor $\bn{E}_H = \ln (\bn{U})$ for the geometric center $\bm{X}_c$ of the HGC\footnote{Note that the log of a tensor is defined by writing it spectrally and replacing each eigenvalue with the log of that eigenvalue. For a case of no shear/off-diagonal terms, you can just take the log of each element on the diagonal to get $\ln(\bn{U})$.}. 
% What are the maximum and minimum values of the strain eigenvalues $E_i(t)$ and $E_i^H(t)$? 
% Would you expect one set to be more symmetric about zero as $\alpha$ gets large, and why?

% \medskip
% (d) Determine both the material point acceleration $\bm{A}(\bm{X}_1)$ at \textcolor{red}{\sout{, and spatial acceleration $\bm{a}(\bm{x}_1)$ of material moving through,}} a point $\frac{1}{2} \bm{e}_1 + 2\bm{e}_2 + \frac{1}{2} \bm{e}_3$.  

% \begin{figure}
% \centering
% \animategraphics[loop,autoplay,width=4in]{10}{instr-figures/The_Happy_Gelatinous_Cube-}{1}{10}
% \end{figure}

% This is a placeholder for the example problems from the first problem set. 
% You'll replace this file with the one I supply on canvas. 
\setcounter{section}{1} 
\section*{Solutions to Examples II: Kinetics, Constitutive Laws, and Viscoelasticity I}
\label{soln-PS2}

\medskip
\subsection*{2--1. \textbf{Balance of mass} [4 pts].} 
We can start with the general form of the conservation of mass equation:
\begin{equation}
\label{eq:continuity}
    \frac{\partial \rho}{\partial t} + \bm{\nabla}_{\bm{x}} \rho \cdot\bm{v} + \rho \bm{\nabla}_{\bm{x}} \cdot \bm{v} = 0.
\end{equation}
    Now we want to incorporate the gradient in spherical coordinates, which will give us a few terms for each of the second and third terms here. 
    The gradient of a scalar function in spherical coordinates is:
\begin{equation*}
    \bm{\nabla}_{\bm{x}} \rho = \frac{\partial \rho}{\partial r} \bm{e}_r + \frac{1}{r} \frac{\partial \rho}{\partial \theta} \bm{e}_\theta  + \frac{1}{r\sin\theta}\frac{\partial \rho}{\partial \phi} \bm{e}_\phi  
\end{equation*}
The divergence of a vector function $\bm{v}$ is:
\begin{equation*}
    \bm{\nabla}_{\bm{x}} \cdot \bm{v} = \frac{\partial v_r}{\partial r} + 2\frac{v_r}{r}+ \frac{1}{r}\frac{\partial v_\theta}{\partial \theta} + \frac{1}{r \sin\theta}\frac{\partial v_\phi}{\partial \phi} + \frac{v_\theta}{r}\cot\theta.
\end{equation*}
Now, since $\bm{v} = v_r \bm{e}_r$, we can drop all of the partial derivatives with respect to $\theta$ and $\phi$, which leaves the divergence of velocity as
\begin{equation*}
    \bm{\nabla}_{\bm{x}} \cdot \bm{v} = \frac{\partial v_r}{\partial r} + 2\frac{v_r}{r}.
\end{equation*}
Incorporating incompressibility means that the density $\rho$ cannot be a function of $\bm{x}$ or $t$, so the time derivatives and gradient of $\rho$ corresponding to the first two terms, respectively, in Eq. \ref{eq:continuity} must be zero. 
This leaves
\begin{equation*}
    \frac{\partial v_r}{\partial r} + 2\frac{v_r}{r} = 0,
\end{equation*}
which we can separate and integrate from the bubble wall to a position with a corresponding velocity as follows:
\begin{align*}
    \frac{\partial v_r}{\partial r} &= -2\frac{v_r}{r}\\
    \int_{\dot{R}}^{v_r} \frac{1}{\tilde{v}_r} d\tilde{v}_r &= -2\int_R^r \frac{1}{\tilde{r}} d\tilde{r}\\
    \ln\left(\frac{v_r}{\dot{R}}\right) &= -2 \ln\left(\frac{r}{R}\right)\\
    v_r &= \frac{R^2}{r^2}\dot{R} ~.//
\end{align*}


\medskip
\subsection*{2--2. \textbf{Balance of momenta} [4 pts].}

To start, let's write out the balance of linear momentum.
\begin{equation*}
    \int_{\partial\Omega} \bm{t}~dA + \int_{\Omega} \bm{b} ~dV = \int_{\Omega} \rho \bm{a} ~dV.
\end{equation*}
From here, we can calculate each of the individual integrals on the left hand side. 
The first will benefit from the divergence theorem, which can turn it from a surface integral to a simple volumetric one of the divergence of the stress, which is:
\begin{equation*}
    \bm{t} = \bm{n} \cdot \bm{\sigma} \Rightarrow \bm{\sigma} = -\rho_w g x_3 \bn{I}.
\end{equation*}
The stress can then be used in the volume integral:
\begin{equation*}
    \int_{\partial\Omega} \bm{t}~dA  = \int_{\Omega} \bm{\nabla}_{\bm{x}} \cdot \bm{\sigma} ~dV.
\end{equation*}
As the stress is just a function of $x_3$, the dot product with the gradient operator simplifies to a derivative with respect to $x_3$ only, and the integrand evaluates to:
\begin{equation*}
\bm{\nabla}_{\bm{x}} \cdot \bm{\sigma} = \frac{\partial }{\partial x_3} \left( -\rho_w g x_3 \right) = -\rho_w g.
\end{equation*}
Integrating this constant stress divergence over the volume gives us a total integrated traction of $-4\pi\rho_w g R^3/3$. 
The body force integral is a bit more straightforward as it's just the gravitational force, but we should include the gradient in $x_2$, which is $\rho(\bm{x}) = \rho_w \left(1 + \frac{1}{2}\frac{x_2}{R}\right)$. 
We can be clever with our choice of the primary axis of our sphere to coincide with the direction of $\bm{e}_2$, which makes $x_2 = r\cos\phi$ and $x_1$ and $x_3$ some combination of $\pm r\cos\theta\sin\phi$ and $\pm r\sin\theta\sin\phi$. 
If we want both to be positive, we take $x_1 = r\cos\theta\sin\phi$ and $x_3 = r\sin\theta\sin\phi$. 
Integrating the body force in spherical coordinates (more important for later than now) then becomes
\begin{align*}
    \int_{\Omega} \bm{b} ~dV &= \int_0^\pi \int_0^R \int_0^{2\pi} \rho_w \left(1+\frac{1}{2}\cos\phi\right) g~r^2 \sin\phi ~d\theta dr d\phi\\
    &= 2\pi\rho_w g \int_0^\pi \int_0^R  \left(1+\frac{1}{2}\cos\phi\right)~r^2 \sin\phi ~ dr d\phi\\
    &= 2\pi \frac{R^3}{3}\rho_w g \left.\left(-\cos\theta -\frac{\cos^2\theta}{4}  \right)\right|_0^{\pi}\\
    &= \frac{4\pi R^3}{3}\rho_w g
\end{align*}
The body force is the weight due to gravity, which perfectly offsets the traction on the region occupied by the sphere supplied by the displaced water. 
Thus, this is naturally in equilibrium!
Note though, that this is ``for now''. 
If the sphere were to rotate we'd have to change the direction of the gradient, replacing the $\cos\theta/2$ with a different Cartesian direction's representation in spherical. 
More on this in a bit!

Now we want to consider the integrals of the moment terms. 
The integral for the traction becomes:
\begin{equation*}
    \int_{\partial \Omega} \bm{x} \times \bm{t}~dA  = \int_{\partial\Omega} \bm{x} \times (\nabla_{\bm{x}} \cdot \bm{\sigma})~d
    A. 
\end{equation*}
We can apply the divergence theorem and write this in index notation now, and manipulate this further:
\begin{align*}
   & \int_{ \Omega} \epsilon_{ijk} \frac{\partial}{\partial x_m} (x_j \sigma_{mk}) \bm{e}_i ~dV\\
   =& \int_{ \Omega} -\rho_w g \epsilon_{ijk} \frac{\partial}{\partial x_m} (x_j x_3 \delta_{mk}) \bm{e}_i ~dV\\
   =& -\rho_w g \int_{ \Omega}  \epsilon_{ijk} \frac{\partial}{\partial x_k} (x_j x_3) \bm{e}_i ~dV\\
   =& -\rho_w g \int_{ \Omega}  \epsilon_{ij3}x_j \bm{e}_i ~dV\\
    =& -\rho_w g \int_{ \Omega}  \epsilon_{ijk} (\cancelto{0, \textrm{ since } \epsilon_{ijj} = 0}{x_3 \delta_{jk}} + x_j \delta_{3k}) \bm{e}_i ~dV\\
    =& -\rho_w g \int_{ \Omega}  \epsilon_{ij3}x_j \bm{e}_i ~dV\\
    =& -\rho_w g \int_{ \Omega}  (x_2 \bm{e}_1 - x_1 \bm{e}_2) ~dV.
\end{align*}

Now, let's deal with the body force term:
\begin{equation*}
    \int_{ \Omega} \bm{x} \times \bm{b}~dV  = \int_{ \Omega} \bm{x} \times \rho(\bm{x}) g \bm{e}_3~dV.
\end{equation*}
The density function doesn't affect the cross product, which, in index notation, is:
\begin{align*}
    \int_\Omega \rho(\bm{x}) g\epsilon_{ij3} x_j \bm{e}_3 \bm{e}_i dV\\
    =\int_\Omega \rho(\bm{x}) g(x_2 \bm{e}_1 - x_1 \bm{e}_2) dV.
\end{align*}
The sum of the two terms is (which would be zero if we're in rotational equilibrium):
\begin{align*}
    \int_\Omega (\rho(\bm{x}) g - \rho_w g)(x_2 \bm{e}_1 - x_1 \bm{e}_2) dV.
\end{align*}
Now we'll write everything in spherical coordinates as above,
\begin{align*}
   & \int_0^\pi \int_0^{2\pi} \int_0^R \left( \rho_w\left(1+\frac{1}{2}   \cos\phi\right)g-\rho_w g \right) r^2 \sin\phi \left(r\cos\phi \bm{e}_1 - r \cos\theta \sin\phi \bm{e}_2 \right) dr d\theta d\phi\\
   =&\frac{1}{2}\rho_w g \int_0^\pi \int_0^{2\pi} \int_0^R r^3 \cos\phi \sin\phi \left(\cos\phi \bm{e}_1 - \cos\theta \sin\phi \bm{e}_2 \right) dr d\theta d\phi\\
   =&\frac{1}{8}R^4 \rho_w g \int_0^\pi \int_0^{2\pi}  \left(\cos^2\phi \sin\phi \bm{e}_1 - \cos\theta \cos\phi \sin^2 \phi \bm{e}_2 \right)  d\theta d\phi\\
    =&\frac{\pi}{4}R^4 \rho_w g \int_0^\pi  \cos^2\phi \sin\phi \bm{e}_1  d\phi\\
    =&\frac{\pi}{4}R^4 \rho_w g \cdot -\frac{1}{3}\cos^3\phi\Big|_0^\pi \bm{e}_1 \\
    =& \frac{\pi}{6} \rho_w g R^4 \bm{e}_1.
\end{align*}
Note what this means---there's a non-zero moment! 
This moment arises because the horizontal gradient requires a moment to sustain. 
What would happen is a rotation such that the gradient would settle vertically, such that the bottom would be the densest and the top would be the least dense. 

\bigskip
\subsection*{2--3. \textbf{Viscoelastic data} [4 pts].} 

(a) For linear viscoelasticity, we require the material to respond linearly (i.e. stress proportional to strain) regardless of the time, $t$. 
How this slope changes is the function of time we care about, but the slope must be constant over some region of interest---in this case, something like $\varepsilon<1.5\times10^{-3}$. 
This is in contrast to e.g. quasi-linear viscoelasticity (QLV), where a material may have a non-linear stress-strain response but this functional form $f(\varepsilon)$ must not change based on time. 

(b) The strain data at 100 Pa and 250 Pa are:
\begin{table}[h]
    \caption{Strain values for different stress levels. All strains are $\times 10^{-3}$.}
    \centering
\begin{tabular}{c c c}
     & \textbf{$\bm{\sigma_1=}$ 100 Pa}  & \textbf{$\bm{\sigma_2=}$ 250 Pa}\\ \hline
      \textbf{Time} & \textbf{Strain}  & \textbf{Strain}\\ \hline
    2 & 0.45 & 1.20\\
    5 & 0.53 & 1.45\\
    10 & 0.68 & 1.80\\
    20 & 0.84 & 2.40\\
    40 & 1.00 & 3.00
\end{tabular}
\end{table}

The creep compliance $J_c(t)$ at any time will be $J_c(t) = \epsilon(t)/\sigma_0$. 
Rewriting the tables as creep compliance, we get the following.
\begin{table}[H]
    \caption{Creep compliance $J_c(t)$ at stress levels of 100 Pa and 250 Pa.}
    \centering
\begin{tabular}{c c c}
    & \textbf{$\bm{\sigma_1=}$ 100 Pa}  & \textbf{$\bm{\sigma_2=}$ 250 Pa}\\ \hline
   \textbf{Time} & \textbf{Compliance}  & \textbf{Compliance}\\ \hline
    2 & 0.45 $\times 10^{-5}$ & 0.48 $\times 10^{-5}$\\
    5 & 0.53 $\times 10^{-5}$ & 0.58 $\times 10^{-5}$\\
    10 & 0.68 $\times 10^{-5}$ & 0.72 $\times 10^{-5}$\\
    20 & 0.84 $\times 10^{-5}$ & 0.96 $\times 10^{-5}$\\
    40 & 1.00 $\times 10^{-5}$ & 1.20 $\times 10^{-5}$
\end{tabular}
\end{table}
We now need to figure out the creep compliance function. 
The creep compliance ought to increase as a function of time as the material softens. 
We may not necessarily know the functional form of this response, but we expect a stiff baseline behavior and subsequent relaxation. 
We probably expect it to also plateau to some other value at long times (that we did not measure). 

Other than that, I'd suggest trying something with an exponential, e.g. fitting to $J_c(t) = J_\infty(1-\exp(-t/\tau))$, which is the Kelvin-Voigt material model.
 To do this, we just divide by $J_
\infty$ and take a log of the expression:
\begin{equation*}
     \ln \left(1-\frac{J_c(t)}{J_\infty}\right) = -t/\tau.
\end{equation*}
There'd be a considerable assumption using $J_\infty$ as the value at 40 seconds, but it may be usable for determining the relaxation regardless. 
The result is a relaxation time on the order of $\approx 4-10$, monotonically increasing. 
What we can take away from this is that the relaxation time actually continues to increase as the material is loaded; this suggests either a spectrum of relaxation, some other functional dependence, or that we're missing complexity by using this model.

\bigskip
\subsection*{2--4. \textbf{Impulsive stresses} [4 pts].}

(a) We start with the general form for strain as a function of stress in a viscoelastic material:
\begin{equation*}
    \varepsilon(t) = \int_{0^-}^tJ_c(t-\tau) \dot{\sigma}(\tau) d\tau.
\end{equation*}
This is okay for the first part of our problem once we substitute in our actual stress function:
\begin{equation*}
     \varepsilon(t) = \int_{0^-}^tJ_c(t-\tau)  A \psi(\tau) d\tau.
\end{equation*}
Then, we need to use a specific $J_c(t)$ for the second half. 
The Kelvin-Voigt function for $J_c(t)$ is
\begin{equation*}
    J_c(t) = \frac{1}{E} \left( 1 - e^{-Et/\eta}\right).
\end{equation*}
This will be most straightforward to approach in Laplace space.
\begin{equation*}
    J(t)*\dot{\sigma}(t) = \bar{J}(s)\cdot \bar{\dot{\sigma}}(s) =  s\bar{{\sigma}}(s)\bar{J}(s).
\end{equation*}

We just need some transforms now.
\begin{align*}
    \bar{\varepsilon}(s) &= \frac{1}{E} \left( \frac{1}{s} - \frac{1}{s+\frac{E}{\eta}}\right) \cdot s\cdot A\\
    &= \frac{A}{E}\left( 1 - \frac{s}{s+\frac{E}{\eta}} \right)\\
    \varepsilon(t) &= \frac{A}{\eta} e^{-Et/\eta}.
\end{align*}


(b) We follow up with a doublet stress, where $\sigma(t) = B\psi(t)$. 
\begin{align*}
    \bar{\varepsilon}(s) &= \frac{1}{E} \left( \frac{1}{s} - \frac{1}{s+\frac{E}{\eta}}\right) \cdot s\cdot Bs\\
    &= \frac{B}{E}\left( s - \frac{s^2}{s+\frac{E}{\eta}} \right)\\
    \varepsilon(t) &= \frac{B}{E}\delta(t) - \frac{BE}{\eta^2} e^{-Et/\eta}.
    %\varepsilon(t) &=\frac{B}{\eta}\psi(t) - \frac{B E}{\eta^2}\delta(t) +  \frac{BE^2}{\eta^3} e^{-Et/\eta}.
\end{align*}
Note that this will converge to $\varepsilon(t>0^+)=- \frac{BE}{\eta^2} e^{-Et/\eta}$ after the initial delta function. 


\bigskip
\subsection*{2--5. \textbf{Two-element models} [8 pts].}

(a+b) The differential constitutive law for a Kelvin-Voigt material is
\begin{equation*}
    \eta \dot{\varepsilon}(t) + E \varepsilon(t) = \sigma(t). 
\end{equation*}
If we take the strain to be $\varepsilon(t) = -\varepsilon_0 \mathcal{H}(t) - \varepsilon_0 \sin\omega t$, we just need to take a derivative of the strain function to get the strain rate: $\dot{\varepsilon}(t) = -\omega \varepsilon_0 \cos\omega t + \varepsilon_0\delta(t)$.
We can then get the stress,
\begin{align*}
    \sigma(t) &= \omega \eta (-\varepsilon_0\cos\omega t - \varepsilon_0 \delta(t)) + E(-\varepsilon_0 - \varepsilon_0 \mathcal{H}(t) \sin\omega t).
\end{align*}
From here, we can use an identity to combine the cosine and sine terms:
\begin{equation*}
     A \sin \omega t + B \cos \omega t= \sqrt{A^2 + B^2} \sin (\omega t + \delta), \textrm{~~where } \delta=\tan^{-1}\frac{B}{A}.
\end{equation*}
Rearranging the stress terms and considering times $t>0^+$, we get:
\begin{equation*}
    \sigma(t) = -E\varepsilon_0 - \varepsilon_0 \sqrt{\omega^2 \eta^2 + E^2} \sin\left( \omega t + \tan^{-1} \frac{\omega \eta}{E}\right).
\end{equation*}
Thus, $\tan \delta = \omega\eta/E$.

(c+d) Now, we'll consider the case where we prescribe the stress function $\sigma(t)$. 
The stress here will be $\sigma(t) = -\sigma_0 \mathcal{H}(t) - \sigma_0 \sin \omega t$, which leads to a differential form of: 
\begin{align*}
     \eta \dot{\varepsilon}(t) + E \varepsilon(t) = -\sigma_0  - \sigma_0 \sin \omega t. 
\end{align*}
Now, divide both sides by $\eta$ and solve this using integrating factors.
\begin{align*}
    \dot{\varepsilon} e^{E t/\eta} + \frac{E}{\eta}\varepsilon e^{E t/\eta} &= \frac{1}{\eta}e^{Et/\eta}\left[-\sigma_0 - \sigma_0 \sin \omega t\right]\\
    \frac{d}{dt}\left( \varepsilon e^{Et/\eta} \right) &= \frac{1}{\eta}e^{Et/\eta}\left[-\sigma_0 - \sigma_0 \sin \omega t\right]\\
    (\varepsilon(t) - \varepsilon(0^-))e^{Et/\eta} &= \int_{0^-}^t \frac{1}{\eta}e^{E\tilde{t}/\eta}\left[-\sigma_0 - \sigma_0 \sin \omega \tilde{t}\right] d\tilde{t}.
\end{align*}
I'll note here that the integral of an exponential multiplied by a sine function gives:
\begin{equation*}
    \int e^{at} \sin \omega t dt = \frac{e^{at}}{a^2 + \omega^2}(a \sin \omega t - \omega \cos \omega t) + \mathcal{C},
\end{equation*}
giving us:
\begin{align*}
 (\varepsilon(t) - \varepsilon(0^-))e^{Et/\eta} &= -\frac{\sigma_0}{\eta} \frac{e^{Et/\eta}}{\left( \frac{E}{\eta}\right)^2 + \omega^2}\left(\frac{E}{\eta} \sin \omega t - \omega \cos \omega t\right)\Bigg|_{0^-}^t - \frac{\sigma_0}{E} e^{Et/\eta}\\
 \varepsilon(t) &= -\frac{\sigma_0}{\eta} \frac{1}{\left( \frac{E}{\eta}\right)^2 + \omega^2}\left(\frac{E}{\eta} \sin \omega t - \omega \cos \omega t - \omega e^{-Et/\eta}\right)  - \frac{\sigma_0}{E}. 
\end{align*}
Now we take the exact same approach as above, which leads us to $\tan \delta = -\omega\eta/E$. 
This essentially says that stress lags the strain response by $\delta$ either way; we have kind of defined $\tan \delta$ here at strain with respect to stress, which should be the reverse of stress with respect to strain. 
Thus, $\tan \delta$ is the same for both.



%% For citations use: 
%%       \citet{<label>} ==> Lamport (1994)
%%       \citep{<label>} ==> (Lamport, 1994)
%%
%Example citation, See \citet{lamport94}.

%% If you have bib database file and want bibtex to generate the
%% bibitems, please use
%%
%%  \bibliographystyle{elsarticle-harv} 
%%  \bibliography{<your bibdatabase>}

%% else use the following coding to input the bibitems directly in the
%% TeX file.

%% Refer following link for more details about bibliography and citations.
%% https://en.wikibooks.org/wiki/LaTeX/Bibliography_Management

%\begin{thebibliography}{00}

%% For authoryear reference style
%% \bibitem[Author(year)]{label}
%% Text of bibliographic item

\bibliographystyle{elsarticle-harv} 
\bibliography{cas-refs}

%\end{thebibliography}
\end{document}

\endinput
%%
%% End of file `elsarticle-template-harv.tex'.


